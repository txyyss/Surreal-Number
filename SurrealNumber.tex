\documentclass[cs4size,a4paper,adobefonts]{ctexart}

\usepackage{hyperref}
\usepackage{indentfirst}
\usepackage[a4paper,left=2.5cm,right=2.5cm,bottom=2.5cm,top=2.5cm]{geometry}
\usepackage{fontspec}
\setmainfont{Minion Pro}
\pagestyle{plain}
%\usepackage{unicode-math}
%\setmathfont{STIX Math}

\usepackage{amsmath,amsthm}
\newtheorem{defn}{定义}
\newtheorem{thm}{定理}

\begin{document}
\title{\bfseries Surreal Number 的学习笔记}
\author{王盛颐}
\date{}
\maketitle
\section{缘起}

Surreal Number (下称“超现实数”)是 John Horton Conway 创造的一种数域,
具体历史可以看 Wikipedia 上的解释。我知道这个概念很久了,最早是大二的时
候看到科普名家谈祥柏先生翻译的一本新书《取胜之道》,这书是对 1986 年出
版的两卷本``Winning Ways for Your Mathematical Plays''的翻译,讨论的是
名为``Combinatorial Game Theory''的,关于两人博弈游戏的数学理论,里面用
到了超现实数作为工具。我对这类博弈理论(和经济学里面的博弈论是两回事)
的兴趣也久,关于这方面的书我能买的买,能下载的下载,基本上都收集全了,
却因为兴趣广杂,专注欠缺迟迟而没有开始。最近刚好在看 Knuth 的那本小说
``Surreal Numbers'',于是决定重新开始,按照这本书的脉络学习超现实数的相
关理论,为后面学习 CGT 打下基础,毕竟 better late than never。

\section{超现实数的定义}
\begin{defn}
  每个超现实数 $x$ 都是一对已构造出的超现实数组成的集合,$x=\{X_L \mid
  X_R\}$。$X_L$ 和 $X_R$ 都是已有的超现实数的子集,分别称为 $x$ 的左集和右集。
  $X_R$ 中任意一个数都不\underline{小于或等于} $X_L$ 中任意一个数。
\end{defn}

这是一个递归定义,递归没问题,但有个缺憾,我们不知道对超现实数来讲,什
么叫做“小于或等于”,于是就有了下面这第二个定义:

\begin{defn}
  一个超现实数 $x$ 小于或等于另一个超现实数 $y$,当且仅当 $y$ 不小于或
  等于 $x$ 的左集中任何一数,且 $y$ 的右集中没有数小于或等于 $x$。
\end{defn}

这个定义用符号来记就是说 $x \leq y$ 当且仅当下面两个条件成立:
$$
\{m \mid m\in X_L, y \leq m \}=\emptyset,
\{n \mid n\in Y_R, n \leq x \}=\emptyset
$$ 

有了这两个定义,可以说超现实数的构造方法就已经清楚了。我们可以根据定义
开始构造超现实数,并证明许多有趣的事实。
\end{document}
