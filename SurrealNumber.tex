\documentclass[cs4size,a4paper,adobefonts]{ctexart}

\usepackage[colorlinks=true,linkcolor=black]{hyperref}
\usepackage{indentfirst}
\usepackage[a4paper,left=2.5cm,right=2.5cm,bottom=2.5cm,top=2.5cm]{geometry}
\usepackage{fontspec}
\setmainfont{Minion Pro}
\pagestyle{plain}
\punctstyle{kaiming}
%\usepackage{unicode-math}
%\setmathfont{STIX Math}

\usepackage{amsmath,amsthm,amssymb}
\newtheorem{defn}{定义}
\newtheorem{thm}{定理}
\newcommand{\pname}[1]{\underline{#1}}
\numberwithin{equation}{section}

\begin{document}
\title{\bfseries Surreal Number 的学习笔记}
\author{王盛颐}
\date{}
\maketitle
\section{缘起}

Surreal Number (下称\pname{超现实数})是 John Horton Conway 创造的一种
数域,具体历史可以看 Wikipedia 上的解释。我知道这个概念很久了,最早是大
二的时候看到科普名家谈祥柏先生翻译的一本新书《取胜之道》,这书是对
1986 年出版的两卷本 \textit{Winning Ways for Your Mathematical Plays}
的翻译,讨论的是名为 ``Combinatorial Game Theory'' 的,关于两人博弈游戏
的数学理论,里面用到了超现实数作为工具。我对这类博弈理论(和经济学里面
  的博弈论是两回事)的兴趣也久,关于这方面的书我能买的买,能下载的下载,
基本上都收集全了,却因为兴趣广杂,专注欠缺迟迟而没有开始。最近刚好在看
Knuth 的那本小说 \textit{Surreal Numbers},于是决定重新开始,按照这本书
的脉络学习超现实数的相关理论,为后面学习 CGT 打下基础,毕竟 better
late than never。

\section{超现实数的定义}
\begin{defn}
  \label{defnSurreal}
  每个超现实数 $x$ 都是一对集合,记为 $x=\{X_L \mid X_R\}$。$X_L$ 和
  $X_R$ 都是已构造出的全部超现实数的子集,分别称为 $x$ 的左集和右集。
  $X_R$ 中任意一个数都不\pname{小于等于} $X_L$ 中任意一个数。
\end{defn}

这是一个递归定义,递归没问题,问题是我们还不知道对超现实数来讲,什么叫
做\pname{小于等于},于是就有了下面这个定义:

\begin{defn}
  \label{defnLeq}
  一个超现实数 $x$ 小于等于另一个超现实数 $y$,当且仅当 $y$ 不小于或
  等于 $x$ 的左集中任何一数,且 $y$ 的右集中没有数小于等于 $x$。
\end{defn}

这同样也是一个递归的定义。为了能够把这两个定义看得更清楚些,我们可以先
做一些符号约定:

\newtheorem*{symbolDef}{符号约定}
\begin{symbolDef}
  符号 $\leq$ 表示超现实数定义中的\pname{小于等于},符号 $\nleq$ 表示
  \pname{不小于等于},也就是说 $x \nleq y$ 等价于 $\neg(x \leq y)$。
\end{symbolDef}

有了上面的符号约定,定义~\ref{defnSurreal} 即是说 $x=\{X_L \mid X_R\}$
是一个超现实数,除了满足 $X_L$ 和 $X_R$ 均由现有超现实数组成之外,还需
满足下面的条件:
\begin{equation}
  \label{eqSurreal}
  \forall\, a \in X_L \,\forall\, b \in X_R:\: b\nleq a.
\end{equation}

定义~\ref{defnLeq} 即是说
\begin{equation}
  \label{eqLeq}
  x \leq y \quad\Leftrightarrow\quad
  \neg(\exists\, x_L \in X_L :\: y \leq x_L)\, \wedge \,
  \neg(\exists\, y_R \in Y_R :\: y_R \leq x).
\end{equation}

现在定义有了,符号有了,我们下面做什么呢?或者说,根据这两个看上去空空
如也的定义,能做些什么呢?我们可以试着构造一个超现实数。根据定
义~\ref{defnSurreal},首先要有两个超现实数的集合才能定义超现实数,但现
在似乎什么都没有,无从下手啊。但是:空集也是集合,而且也是已有超现实数
的一个子集!

于是,我们根据空集可以构造出这个东西来:$\{\emptyset\mid\emptyset\}$,
下面不妨简记为 $\{\,\mid\,\}$。检查一下它符合定义~\ref{defnSurreal} 么?
左集右集空无一物,当然满足 \eqref{eqSurreal} 的条件,所以第一个超现实数
就已经有了:$\{\,\mid\,\}$。

我们给第一个超现实数 $\{\,\mid\,\}$ 一个名字,称之为“零”,用符号 $0$ 来
表示,也即
\begin{equation}
  0 \equiv \{\,\mid\,\}
\end{equation}
这里 $\equiv$ 的意思是等同替代,0 只是为了表示 $\{\,\mid\,\}$ 的一个简
记符号,它和我们熟知的自然数 0 只是碰巧符号相同而已。$\equiv$ 和等于号
$=$ 也没有任何联系,我们还不知道超现实数的等于是什么定义呢。

$\{\,\mid\,\}$ 符合定义~\ref{defnSurreal},是我们能构造的第一个超现实数,
然后看定义~\ref{defnLeq},是不是有 $0 \leq 0$ 呢?注意,这里 $0$ 并不是
自然数,所以成立与否还是需要证明的。我们下面开始证明确实有
\begin{equation}
  \label{eqZeroLeqZero}
  \{\,\mid\,\}\leq\{\,\mid\,\}
\end{equation}

\begin{proof}[证明]
  根据 \eqref{eqLeq},若要证明 \eqref{eqZeroLeqZero},就是要证明
  \[
  \neg(\exists\,x_L\in\emptyset:\:\{\,\mid\,\}\leq x_L)\,\wedge\,
  \neg(\exists\,y_R\in\emptyset:\:y_R\leq\{\,\mid\,\})
  \]
  虽然我们还是不知道上面这个命题里 $\leq$ 的定义,但由于空集的原因,这
  个命题显然为真。所以我们证明了 $0\leq0$ 成立。
\end{proof}

\end{document}
