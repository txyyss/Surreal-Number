\documentclass[cs4size,a4paper,adobefonts]{ctexart}
\usepackage{amsmath,amsthm,amssymb}
\usepackage[colorlinks=true,linkcolor=black]{hyperref}
\usepackage{indentfirst}
\usepackage[a4paper,left=2.5cm,right=2.5cm,bottom=2.5cm,top=2.5cm]{geometry}
\usepackage{fontspec}
\setmainfont{Palatino}
\setmonofont[Scale=MatchLowercase]{Monaco}
\pagestyle{plain}
\punctstyle{kaiming}
\usepackage{unicode-math}
\setmathfont{Asana Math}

\newtheorem{defn}{定义}
\newtheorem{thm}{定理}
\newcommand{\pname}[1]{\underline{#1}}
\numberwithin{equation}{section}

\usepackage{fancyvrb}
\usepackage{etoolbox}
\makeatletter
\patchcmd{\FV@ListVSpace}{\@topsepadd\topsep}{}{}
\makeatother
\begin{document}
\title{\bfseries Surreal Number 的学习笔记}
\author{王盛颐}
\date{}
\maketitle
\section*{缘起}

Surreal Number (下称\pname{超现实数})是 John Horton Conway 创造的一种
数域,具体历史可以看 Wikipedia 上的解释。我知道这个概念很久了,最早是大
二的时候看到科普名家谈祥柏先生翻译的一本新书《取胜之道》,这书是对1986
年出版的两卷本 \textit{Winning Ways for Your Mathematical Plays} 的翻译,
讨论的是名为 ``Combinatorial Game Theory'' 的,关于两人博弈游戏的数学理
论,里面用到了超现实数作为工具。我对这类博弈理论(和经济学里面的博弈论
  是两回事)的兴趣也久,关于这方面的书我能买的买,能下载的下载,基本上
都收集全了,却因为兴趣广杂,专注欠缺迟迟而没有开始。最近刚好在看Knuth
的那本小说 \textit{Surreal Numbers},于是决定重新开始,按照这本书的脉络
学习超现实数的相关理论,为后面学习 CGT 打下基础,毕竟 better late than
never。

\section{超现实数的引入}
\begin{defn}
  \label{defn:Surreal}
  每个超现实数 $x$ 都是一对集合,记为 $x=\{X_L \mid X_R\}$。$X_L$ 和
  $X_R$ 都是已构造出的全部超现实数的子集,分别称为 $x$ 的左集和右集。
  $X_R$ 中任意一个数都不\pname{小于等于} $X_L$ 中任意一个数。
\end{defn}

这是一个递归定义,递归没问题,问题是我们还不知道对超现实数来讲,什么叫
做\pname{小于等于},于是就有了下面这个定义:

\begin{defn}
  \label{defn:Leq}
  一个超现实数 $x$ 小于等于另一个超现实数 $y$,当且仅当 $y$ 不小于或
  等于 $x$ 的左集中任何一数,且 $y$ 的右集中没有数小于等于 $x$。
\end{defn}

这同样也是一个递归的定义。为了能够把这两个定义看得更清楚些,我们可以先
做一些符号约定:

\newtheorem*{symbolDef}{符号约定}
\begin{symbolDef}
  符号 $\leq$ 表示超现实数定义中的\pname{小于等于},符号 $\nleq$ 表示
  \pname{不小于等于},也就是说 $x \nleq y$ 等价于 $\neg(x \leq y)$。
\end{symbolDef}

有了上面的符号约定,定义~\ref{defn:Surreal} 即是说 $x=\{X_L \mid X_R\}$
是一个超现实数,除了满足 $X_L$ 和 $X_R$ 均由现有超现实数组成之外,还需
满足下面的条件:
\begin{equation}
  \label{eq:Surreal}
  \forall\, a \in X_L \,\forall\, b \in X_R:\: b\nleq a.
\end{equation}

定义~\ref{defn:Leq} 即是说
\begin{equation}
  \label{eq:Leq}
  x \leq y \quad\Leftrightarrow\quad
  \neg(\exists\, x_L \in X_L :\: y \leq x_L)\, \wedge \,
  \neg(\exists\, y_R \in Y_R :\: y_R \leq x).
\end{equation}

现在定义有了,符号有了,我们下面做什么呢?或者说,根据这两个看上去空空
如也的定义,能做些什么呢?我们可以试着构造一个超现实数。根据定
义~\ref{defn:Surreal},首先要有两个超现实数的集合才能定义超现实数,但现
在似乎什么都没有,无从下手啊。但是:空集也是集合,而且也是已有超现实数
的一个子集!

于是,我们根据空集可以构造出这个东西来:$\{\emptyset\mid\emptyset\}$,
下面不妨简记为 $\{\,\mid\,\}$。检查一下它符合定义~\ref{defn:Surreal} 么?
左集右集空无一物,当然满足 \eqref{eq:Surreal} 的条件,所以第一个超现实数
就已经有了:$\{\,\mid\,\}$。

我们给第一个超现实数 $\{\,\mid\,\}$ 一个名字,称之为“零”,用符号 $0$ 来
表示,也即
\begin{equation}
  0 \equiv \{\,\mid\,\}
\end{equation}
这里 $\equiv$ 的意思是等同替代,0 只是为了表示 $\{\,\mid\,\}$ 的一个简
记符号,它和我们熟知的自然数 0 只是碰巧符号相同而已。$\equiv$ 和等于号
$=$ 也没有任何联系,我们还不知道超现实数的等于是什么定义呢。

$\{\,\mid\,\}$ 符合定义~\ref{defn:Surreal},是我们能构造的第一个超现实数,
然后看定义~\ref{defn:Leq},是不是有 $0 \leq 0$ 呢?注意,这里 $0$ 并不是
自然数,所以成立与否还是需要证明的。我们下面开始证明确实有
\begin{equation}
  \label{eq:ZeroLeqZero}
  \{\,\mid\,\}\leq\{\,\mid\,\}
\end{equation}

\begin{proof}[证明]
  根据 \eqref{eq:Leq},若要证明 \eqref{eq:ZeroLeqZero},就是要证明
  \[
  \neg(\exists\,x_L\in\emptyset:\:\{\,\mid\,\}\leq x_L)\,\wedge\,
  \neg(\exists\,y_R\in\emptyset:\:y_R\leq\{\,\mid\,\})
  \]
  虽然我们还是不知道上面这个命题里 $\leq$ 的定义,但由于空集的原因,这
  个命题显然为真。所以我们证明了 $0\leq0$ 成立。
\end{proof}

现在我们已经知道了一个超现实数 $0$,还知道了它的一个性质 $0\leq0$,这也
是我们从无到有根据定义所能创造的第一个超现实数。有了第一个数,根据定
义~\ref{defn:Surreal} 我们就能创建更多的数了。

为了清晰起见,下面再做个简记约定。比如,若左集是 $\{0\}$ 而右集是
$\emptyset$,那么根据定义我们应该这么记这个“数”(因为还没有证明这是数,
  所以加引号)为:$\{\{0\}\mid\emptyset\}$。但这么记实在太繁琐了,所以
我们除了省掉空集记号 $\emptyset$ 之外,还省掉表示左右集本身的花括号,只
剩表示超现实数的花括号。这样原先这个“数”就可以记成 $\{0\mid\,\}$。

接下来通过把 $0$ 放在左集或右集,我们能再构造出三个数:
\[
\{\,\mid 0\},\{0\mid\,\}\,\text{和}\,\{0\mid 0\}
\]
其中第三个数 $\{0\mid 0\}$ 因为右集的 $0$ 小于等于左集的 $0$,不符合超
现实数的定义,所以不是超现实数。很容易验证得知,另外两个都是超现实数。

首先让我们证明
\begin{equation}
  \label{eq:0Leq1}
  \{\,\mid\,\}\leq\{0\mid\,\}
\end{equation}
\begin{proof}[证明]
  根据定义~\ref{defn:Leq} 的符号化表示 \eqref{eq:Leq},\eqref{eq:0Leq1} 为
  真当且仅当
  \[
  \neg(\exists\, x_L \in \emptyset :\: \{0\mid\,\} \leq x_L)\, \wedge \,
  \neg(\exists\, y_R \in \emptyset :\: y_R \leq \{\,\mid\,\})
  \]
  同样因为空集的关系,上述命题显然成立。
\end{proof}

还可以证明
\begin{equation}
  \label{eq:1NotLeq0}
  \{0\mid\,\}\nleq\{\,\mid\,\}
\end{equation}
\begin{proof}[证明]
  根据符号约定和 \eqref{eq:Leq},这就是要证明
  \begin{equation}
    \label{eq:0Leq0}
    \exists\, x_L \in \{0\} :\: \{\,\mid\,\} \leq x_L
  \end{equation}
  或
  \[
  \exists\, y_R \in \emptyset :\: y_R \leq \{0\mid\,\}
  \]
  根据 \eqref{eq:ZeroLeqZero},\eqref{eq:0Leq0} 成立,故上述命题成立。
\end{proof}

进一步可以证明
\begin{equation}
  \label{eq:1Leq1}
   \{0\mid\,\}\leq \{0\mid\,\}
\end{equation}
\begin{proof}[证明]
  根据 \eqref{eq:Leq},这就是要证明
  \begin{equation}
    \label{eq:1Leq1_1}
    \neg(\exists\, x_L \in \{0\} :\: \{0\mid\,\} \leq x_L)    
  \end{equation}
  且
  \begin{equation}
    \label{eq:1Leq1_2}
    \neg(\exists\, y_R \in \emptyset :\: y_R \leq \{0\mid\,\})
  \end{equation}
  \eqref{eq:1Leq1_2} 因为空集的缘故显然成立,而由 \eqref{eq:1NotLeq0}
  可知 \eqref{eq:1Leq1_1} 也成立,命题得证。
\end{proof}

为了后续讨论的方便,我们再做符号约定如下:
\begin{symbolDef}
  $x\leq y$ 也可写成 $y\geq x$,符号 $\geq$ 读作\pname{大于等于}。它的
  否定\pname{不大于等于},用符号 $\ngeq$ 表示。

  $x\leq y\wedge y\nleq x$ 可简写为 $x<y$,符号 $<$ 读作\pname{小于}。
  它的否定\pname{不小于},用符号 $\nless$ 表示。

  $x<y$ 也可写成 $y>x$,符号 $>$ 读作\pname{大于}。它的否定\pname{不大
    于},用符号 $\ngtr$ 表示。

  $x\leq y \wedge y \leq x$ 可简写为 $x=y$,符号 $=$ 读作\pname{等于}。
  它的否定\pname{不等于},用符号 $\neq$ 表示。
\end{symbolDef}

这些符号及其意义对于实数来讲可说是平平无奇,但对于超现实数来说,这些符
号现在才有了定义,千万注意不要把关于实数序关系不加证明的简单带入超现实
数中。

有了这些新符号我们现在可以把 \eqref{eq:ZeroLeqZero} 写作
\begin{equation*}
  0=0.
\end{equation*}
而根据 \eqref{eq:0Leq1} 和 \eqref{eq:1NotLeq0},则有
\begin{equation*}
  0 < \{0\mid\,\}
\end{equation*}
同样的,根据 \eqref{eq:1Leq1} 有
\begin{equation*}
  \{0\mid\,\}=\{0\mid\,\}
\end{equation*}

和上述证明步骤类似的,我们很容易可以证明:
\begin{align*}
  \{\,\mid 0\} &< 0\\
  \{\,\mid 0\} &= \{\,\mid 0\}\\
  \{\,\mid 0\} &< \{0\mid \,\}
\end{align*}

现在可以给 $\{0\mid \,\}$ 和 $\{\,\mid 0\}$ 各起一个适当的名字了:我们
把 $\{0\mid \,\}$ 叫做\pname{一},用符号 $1$ 表示;把 $\{\,\mid 0\}$ 叫
做\pname{负一},用符号 $-1$ 来表示:
\begin{align*}
  1 &\equiv \{0\mid \,\}\\
  -1 &\equiv \{\,\mid 0\}
\end{align*}
这样上述我们证明过的事实就可以用符号重新写成:
\begin{align*}
  0 & < 1\\
  1 & = 1\\
  -1 & < 0\\
  -1 & = -1\\
  -1 & < 1.
\end{align*}
这刚好和自然数里 $0,1,-1$ 的序关系是一样的。

我们可以注意到,上述证明都只是在根据定义 \ref{defn:Surreal} 和定义
\ref{defn:Leq} 进行机械的验证。这两个定义本身就是两个递归的验证算法。事
实上,用 Haskell 语言可以很简单的编写出这样的验证程序来。

首先可以定义超现实数类型 \verb|Surreal| 如下:
\DefineVerbatimEnvironment{code}{Verbatim}{gobble=0,xleftmargin=2ex,xrightmargin=2ex}
\fvset{gobble=0,xleftmargin=2ex,xrightmargin=2ex}
\begin{code}
data Surreal = N [Surreal] [Surreal] deriving Show
\end{code}
而表示小于等于的函数 \verb|leq|,验证是否是超现实数的函数 \verb|valid|
则可以这么定义:

\VerbatimInput{code/LeqAndValid.hs}

这两个函数定义其实就是对 \eqref{eq:Leq} 和 \eqref{eq:Surreal} 的直接翻
译。而根据已有的符号约定则可将 \verb|Surreal| 类型的等于和序关系定义如
下:
\begin{code}
instance Eq Surreal where
  x == y = x `leq` y && y `leq` x
instance Ord Surreal where
  (<=) = leq
\end{code}
那么到目前为止,我们已经得到的三个超现实数可以在程序中定义为:
\begin{code}
zero = N [] []
one = N [N [] []] []
minusOne = N [] [N [] []]
\end{code}

有了上述函数和常量的定义,我们就可以跑一些命令试试看了:
\begin{code}
*Surreal> map valid [zero, one, minusOne]
[True,True,True]
*Surreal> zero == zero
True
*Surreal> zero <= minusOne
False
\end{code}
不需要证明,程序本身就能算出命题的真假来,多么神奇。进一步的,我们还可
以用程序算出根据已经有的这三个超现实数 $0,1,-1$ 我们还可以生成哪些超现
实数来。

为了程序输出结果的可读,让我们先定义一个 \verb|Surreal| 的符号表示函数:
\begin{code}
symbolForm :: Surreal -> String
symbolForm (N [] []) = "0"
symbolForm (N [N[] []] []) = "1"
symbolForm (N [] [N[] []]) = "-1"
symbolForm (N ls rs) = "{" ++ listForm ls ++ "|" ++ listForm rs ++ "}"
  where listForm = intercalate "," . map symbolForm
\end{code}
而根据已有超现实数生成新的超现实数的函数则可简单定义如下:
\begin{code}
constructNext :: [Surreal] -> [Surreal]
constructNext x = filter valid [N a b | a <- subsequences x, 
                                b <- subsequences x]
\end{code}
于是,为了得到根据 $0,1,-1$ 可生成的超现实数,只要运行下列命令:
\VerbatimInput{code/GenerateFrom-101.hs}
结果如下:
\begin{code}
["0","{|-1}","-1","{|-1,0}","{|1}","{|-1,1}","{|0,1}","{|-1,0,1}",
"{-1|}","{-1|0}","{-1|1}","{-1|0,1}","1","{0|1}","{-1,0|}",
"{-1,0|1}","{1|}","{-1,1|}","{0,1|}","{-1,0,1|}"]
\end{code}

除掉已有的 $0,1,-1$ 可得到 17 个超现实数。但这 17 个超现实数未必是“新”
的,它们可能等于 $0,1,-1$,这点我们只需要算一下就可以知道情况了:
\VerbatimInput{code/Filter-101.txt}

甚至更进一步,我们可以把已有的这 20 个超现实数的顺序关系全部算出来:
\VerbatimInput{code/OrderFirstGen.txt}
上述运行结果用数学语言即是意味着
\begin{equation}
  \label{eq:firstChain}
  \begin{split}
    \{\,\mid-1\} &=\{\,\mid-1,0\}=\{\,\mid-1,1\}=\{\,\mid-1,0,1\} \\
    &< -1 =\{\,\mid0,1\}\\
    &< \{-1\mid0\}=\{-1\mid0,1\}\\
    &< 0=\{\,\mid1\}=\{-1\mid\,\}=\{-1\mid1\}\\
    &< \{0\mid1\}=\{-1,0\mid1\}\\
    &< 1=\{-1,0\mid\,\}\\
    &< \{1\mid\,\}=\{-1,1\mid\,\}=\{0,1\mid\,\}=\{-1,0,1\mid\,\}
  \end{split}
\end{equation}

上述 \eqref{eq:firstChain} 里的每个小于号和等号都可以严格的证明,但这里
限于篇幅就不证了。程序能做的事情就不用麻烦人类了,这才是程序的意义啊。

\section{基本性质}
\end{document}
