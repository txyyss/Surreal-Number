\documentclass[cs4size,a4paper,adobefonts]{ctexart}
\usepackage{hyperref}
\usepackage{indentfirst}
\usepackage[a4paper,left=2.5cm,right=2.5cm,bottom=2.5cm,top=2.5cm]{geometry}
\usepackage{fontspec}
\setmainfont{Minion Pro}
\pagestyle{plain}
\usepackage{unicode-math}
\setmathfont{STIX Math}
\begin{document}
\title{\bfseries Surreal Number 的学习笔记}
\author{王盛颐}
\date{}
\maketitle
\section{缘起}

Surreal Number (下称“超现实数”)是 John Horton Conway 创造的一种数域,
具体历史可以看 Wikipedia 上的解释。我知道这个概念很久了,最早是大二的时
候看到科普名家谈祥柏先生翻译的一本新书《取胜之道》,这书是对 1986 年出
版的两卷本``Winning Ways for Your Mathematical Plays''的翻译,讨论的是
名为``Combinatorial Game Theory''的,关于两人博弈游戏的数学理论,里面用
到了超现实数作为工具。我对这类博弈理论(和经济学里面的博弈论是两回事)
的兴趣也久,关于这方面的书我能买的买,能下载的下载,基本上都收集全了,
却因为兴趣广杂,专注欠缺迟迟而没有开始。最近刚好在看 Knuth 的那本小说
``Surreal Numbers'',于是决定重新开始,按照这本书的脉络学习超现实数的相
关理论,为后面学习 CGT 打下基础。

\section{定义}


\end{document}
