\documentclass[cs4size,a4paper,adobefonts]{ctexart}

\usepackage[colorlinks=true,linkcolor=black]{hyperref}
\usepackage{indentfirst}
\usepackage[a4paper,left=2.5cm,right=2.5cm,bottom=2.5cm,top=2.5cm]{geometry}
\usepackage{fontspec}
\setmainfont{Minion Pro}
\pagestyle{plain}
%\usepackage{unicode-math}
%\setmathfont{STIX Math}

\usepackage{amsmath,amsthm,amssymb}
\newtheorem{defn}{定义}
\newtheorem{thm}{定理}
\newcommand{\pname}[1]{\underline{#1}}

\begin{document}
\title{\bfseries Surreal Number 的学习笔记}
\author{王盛颐}
\date{}
\maketitle
\section{缘起}

Surreal Number (下称\pname{超现实数})是 John Horton Conway 创造的一种
数域,具体历史可以看 Wikipedia 上的解释。我知道这个概念很久了,最早是大
二的时候看到科普名家谈祥柏先生翻译的一本新书《取胜之道》,这书是对
1986 年出版的两卷本 \textit{Winning Ways for Your Mathematical Plays}
的翻译,讨论的是名为 \textit{Combinatorial Game Theory} 的,关于两人博
弈游戏的数学理论,里面用到了超现实数作为工具。我对这类博弈理论(和经济
  学里面的博弈论是两回事)的兴趣也久,关于这方面的书我能买的买,能下载
的下载,基本上都收集全了,却因为兴趣广杂,专注欠缺迟迟而没有开始。最近
刚好在看 Knuth 的那本小说 \textit{Surreal Numbers},于是决定重新开始,
按照这本书的脉络学习超现实数的相关理论,为后面学习 CGT 打下基础,毕竟
better late than never。

\section{超现实数的定义}
\begin{defn}
  \label{defnSurreal}
  每个超现实数 $x$ 都是一对集合,记为 $x=\{X_L \mid X_R\}$。$X_L$ 和
  $X_R$ 都是已构造出的超现实数形成的集合,分别称为 $x$ 的左集和右集。
  $X_R$ 中任意一个数都不\pname{小于等于} $X_L$ 中任意一个数。
\end{defn}

这是一个递归定义,递归没问题,问题是我们还不知道对超现实数来讲,什么叫
做\pname{小于等于},于是就有了下面这个定义:

\begin{defn}
  \label{defnLeq}
  一个超现实数 $x$ 小于等于另一个超现实数 $y$,当且仅当 $y$ 不小于或
  等于 $x$ 的左集中任何一数,且 $y$ 的右集中没有数小于等于 $x$。
\end{defn}

这同样也是一个递归的定义。为了能够把这两个定义看得更清楚些,我们可以先
做一些符号约定:

\newtheorem*{symbolDef}{符号约定}
\begin{symbolDef}
  符号 $\leq$ 表示超现实数定义中的\pname{小于等于},符号 $\nleq$ 表示
  \pname{不小于等于},也就是说 $x \nleq y$ 等价于 $\neg(x \leq y)$。
\end{symbolDef}

有了上面的符号约定,定义~\ref{defnSurreal} 即是说 $x=\{X_L \mid X_R\}$
是一个超现实数,除了满足 $X_L$ 和 $X_R$ 均由现有超现实数组成之外,还需
满足下面的条件:
$$
\forall a \in X_L \,\forall b \in X_R: b\nleq a.
$$

定义~\ref{defnLeq} 即是说 $x \leq y$ 当且仅当下面两个条件成立:
$$
\nexists\, m \in X_L, \text{s.t. }y \leq m \quad
\nexists\, n \in Y_R, \text{s.t. }n \leq x.
$$
 
\end{document}
